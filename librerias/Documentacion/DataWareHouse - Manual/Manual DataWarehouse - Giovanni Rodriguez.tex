\documentclass[12pt, letterpaper]{article}
\usepackage[utf8]{inputenc}
\usepackage[spanish]{babel}

\begin{document}
\title{Proyecto Data Warehouse}
\author{Héctor Giovanni Rodríguez Ramos}
\date{20 de Enero de 2014}
\maketitle
\begin{center}
Documentación de la programación en lenguaje de programación Java empleada en el proyecto Data Warehouse.
\end{center}

\newpage
Definiciones:\\
Data Warehouse: Configuración de un sistema en el cual se acumula toda la información relativa a una organización. Este sistema contempla hardware y software que administra los datos de diversas fuentes de una organización.\\
Por ejemplo, sea una empresa y sus diferentes departamentos, el departamento A maneja información que el departamento B trabaja, lógicamente el departamento A solicita la información directamente al departamento B, pero esto no es optimo y se pueden producir muchas inconsistencias ya que la información cambia a través del tiempo.\\
Una solución consiste en mantener centralizada la información en la organización mediante un conjunto de normas y restricciones de acceso, los cuales aseguran que la información será correcta.\\



\end{document}